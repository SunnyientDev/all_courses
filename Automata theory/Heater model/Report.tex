\documentclass[
aps,%
12pt,%
final,%
notitlepage,%
oneside,%
onecolumn,%
nobibnotes,%
nofootinbib,%
superscriptaddress,%
noshowpacs,%
centertags]%
{revtex4}
\usepackage{graphicx,times}
%\usepackage{natbib}
%\usepackage{lscape}
%\usepackage{tgtermes}
\usepackage{longtable}
\usepackage{amssymb,amsmath}
\usepackage{caption2}
\usepackage{xcolor}
\usepackage{multirow}

\newcommand{\hii}    {H\,{\sc{ii}}}

%Russian-specific packages
%----------------------------
\usepackage[T2A]{fontenc}
\usepackage[utf8]{inputenc}
\usepackage[russian]{babel}

% for-formulas
\usepackage{amssymb,amsmath,amsthm}
% \usepackage[dutch]{babel}
\usepackage{systeme,mathtools}
\makeatletter
\renewcommand*\env@matrix[1][*\c@MaxMatrixCols c]{%
  \hskip -\arraycolsep
  \let\@ifnextchar\new@ifnextchar
  \array{#1}}
\makeatother
\usepackage{lipsum}
\usepackage{relsize}
\newcommand\md{\ }

%\bibpunct{(}{)}{;}{a}{}{,}

\newcommand{\todo}[1]{\textcolor{red}{\textbf{TODO}~#1}}
\newcommand{\nasty}[2]{\textcolor{blue}{\textbf{nasty}~#1}}
\usepackage[normalem]{ulem}
\input{sao_cmd.tex}


% Начало документа

\begin{document}
\selectlanguage{russian}

\title{\textcolor{blue}{Создание свёрточной нейронной сети для п}\sout{П}оиска \sout{кандидатов в скоплений галактик} \textcolor{red}{эффекта Сюняева-Зельдовича} на картах микроволнового фонового излучения космической миссии Planck\sout{ с помощью свёрточной нейронной сети}}


\author{А.~Д.~Ороновская}
\affiliation{Астрофизическая школа ``Траектория'', Россия}

\author{О.~В.~Верходанов}
%\email{vo@sao.ru}
\affiliation{Специальная астрофизическая обсерватория РАН, Нижний Архыз, Карачаево-Черкесия, Россия}

\author{А.~П.~Топчиева}
\email{ATopchieva@inasan.ru}
\affiliation{Институт астрономии РАН, Москва, Россия}

\author{Д.~А.~Шорин}
\affiliation{Астрофизическая школа ``Траектория'', Россия}

\author{С.~А.~Базров}
\affiliation{Астрофизическая школа ``Траектория'', Россия}

%\date{\today}
%\today печатает сегодняшнее число

\begin{abstract}
В работе предложен метод поиска \sout{кандидатов в скоплений галактик} \textcolor{red}{объектов} с эффектом Сюняева-Зельдовича (далее, "СЗ-эффект'') с помощью сверточной нейронной сети. Используемый метод машинного обучения \sout{для поиска СЗ-эффекта, как предполагается} позволит ускорить селекцию кандидатов в скопления галактик с \textcolor{red}{СЗ-}эффектом.

\sout{В каталоге обучения сети содержатся произволится по данные об объектах с СЗ-эффектом миссии телескопа Planck, которые представлены в виде изображений, сделанных на частотах: 100, 143, 217, 353 и 545 ГГц, а к} \textcolor{red}{К}аталог для распознавания \sout{скоплений галактик} \textcolor{red}{радиоисточников} составлен с помощью схемы пикселизации GLESP (Gauss-Legendre Sky Pixelization) \textcolor{blue}{на частотах: 100, 143, 217, 353 и 545 ГГц}. \sout{Использование методов машинного обучения для поиска СЗ-эффекта, как предполагается, позволит ускорить селекцию кандидатов в скопления галактик с эффектом.} \textcolor{blue}{Показано, что данный подход может быть применен к анализу массивов данных, что позволит по ускоренной процедуре отбирать наиболее вероятные кандидаты в скопления галактик на больших красных смещениях.}
\end{abstract}

\maketitle
\section{Введение}
\label{sec:intro}

Одним из результатов работы космического телескопа Planck является создание карты распределения интенсивности реликтового излучения для всего неба. В числе прочего эта карта включает в себя объекты с эффектом Сюняева-Зельдовича (далее СЗ-эффект), который можно использовать для идентификации скоплений галактик.\sout{В 2018 году представлен последний релиз данных космической миссии Planck \footnote{\tt https://pla.esac.esa.int/#home} Европейского космического агентства. Объем информации, доступной для научного анализа, составляет несколько терабайт и требует автоматической обработки для поиска и отождествления объектов с \sout{заданными характеристикам}. При этом часть наблюдаемых объектов на картах неба пропущена в публикуемых каталогах. Однако эта проблема может быть решена при помощи новых алгоритмов и программ, чувствительных к топологическим, статистическим и спектральным характеристикам многочастотных карт космических миссий.}
\textcolor{red}{Эффект-СЗ является результатом рассеяния фотонов реликтового излучения на горячих электронах в плазме. Подходящие условия для его наблюдения присутствуют в скоплениях галактик~\cite{zs}). Согласно космологическим исследованиям, изучение таких скоплений при различных красных смещениях может помочь нам понять устройство нашей Вселенной, в результате проверки космологических моделей и ограничения космологических постоянных~\cite{Allen, Kravtsov, Nagai}. Кроме того, такие объекты могут помочь исследовать и другие аспекты, касаемые не только космологии, но и астрофизики~\cite{Sarazin, Kormendy, Bykov}.} 
Среди лидирующих направлений в космологических исследованиях остается изучение скоплений галактик в миллиметровом и субмиллиметровом диапазонах, наблюдаемых благодаря СЗ-эффекту \cite{zs}, в рентгеновском диапазоне, где излучает горячий газ, и в видимом свете. Эти исследования позволяют проследить эволюцию масс скоплений и особенности формирования крупномасштабных структур Вселенной в разные космологические эпохи\textcolor{red}{~\cite{Basu}}. 

\sout{Свойство этого эффекта заключается в том, что спектр и яркость в направлениях на скопления не зависят от расстояния, на котором находится скопление, и по свойствам многочастотного микроволнового излучения можно осуществить поиск скоплений галактик измерить ряд физических параметров (например, температуру микроволнового фона в этом направлении, по которой можно оценить массу скопления).} 

\sout{Текущие исследования скоплений галактики с помощью СЗ-эффекта связаны с поиском и пополнением выборки кандидатов, изучением генерирования и взаимодействия энергичных электронов с фотонами реликтового излучения в джетах радиогалактик и окрестностях гигантских эллиптических галактик. Кроме того, объекты с СЗ-эффектом могут использоваться для независимого определения космологических параметров при наличии достаточной статистики. Это направление поможет получить оценки космологических параметров, таких как постоянная Хаббла, параметры плотности и отделения различных компонентов в угловой статистике флуктуаций фона \cite{Barbosa1996,planck_zs}. СЗ-эффект применяется в симуляциях формирования гидродинамической структуры и анализа данных о тепловых и кинетических эффектах в теории~\cite{2009MNRAS.397L..41C} и для оценки вклада в микроволновое фоновое излучение неба на минутных угловых масштабах \cite{dsol_model}. }

\sout{Расширение списка} \textcolor{red}{В данной работе мы хотим представить алгоритм машинного обучения, который мог бы помочь расширить список} скоплений галактик в миллиметровом диапазоне. \sout{связано с появлением многочастотных измерений микроволнового излучения, подобных экспериментам Planck \cite{planck_zs}, SPT \cite{spt_zs} и ACT \cite{act_zs}. Первые данные обсерватории Planck показали, что зарегистрированное число скоплений галактик ($\sim$1.6\,тыс.), наблюдаемых с помощью СЗ-эффекта, на 2 порядка меньше \cite{Planck1, Planck2}, чем ожидается по данным оптических обзоров и моделированию. Селекционные эффекты, фоновое излучение нашей Галактики, точечные источники излучения, чей вклад в микроволновый фон перекрывает глубину CЗ-эффекта, а также зависимость амплитуды излучения (определяется этим эффектом) от массы скоплений могут влиять на результаты обнаружения скоплений галактик с СЗ-механизмом.}\textcolor{red}{Есть данные наблюдений микроволнового фона Planck \cite{planck_zs}, SPT \cite{spt_zs} и  других обсерваторий, например ACT \cite{act_zs}. Эти наблюдения представляют собой карты неба на различных длинах волн. Эффект наблюдается на некоторых участках карты (в направлении на скопления) и имеет особую спектральную форму: на частотах 217~ГГц наблюдается повышение интенсивности над микроволновым фона, а на более высоких частотах --- понижение. Необходимо найти такие участки на карте.}

Наблюдения эффекта трудоемки из-за его малой амплитуды, экспериментальной погрешности и искажения температуры реликтового излучения другими источниками. Поскольку СЗ-эффект --- эффект рассеяния, и его величина не зависит от красного смещения, скопления с высоким красным смещением могут быть обнаружены так же легко, как и при низком красном смещении, если у них достаточная масса ($M_{\rm clust}\sim10^{14} M_{\rm sun} $).
Отношение угловой шкалы к красному смещению также является фактором, облегчающим обнаружение скопления с высоким красным смещением: оно мало меняется между красными смещениями 0.3 и 2, что означает, что скопления между этими красными смещениями имеют практически одинаковые размеры на небе.

\sout{Дополнительная информация, которая поможет ускорить поиск далеких скоплений галактик, связана с исследование радиоисточников \cite{blum_miley,rg_list1,rg_book}. Физические свойства радиоисточников позволяют сделать их мощным средством при тестировании космологических эпох, поскольку с ними связаны поиски самых далеких активных ядер галактик \cite{par_big3a,uss_list} и протоскоплений \cite{rg_protocl}, оценка скучивания фоновых объектов на разных красных смещениях \cite{rg_protocl,tim_clust}, исследования гравитационного линзирования. С учетом возможностей миллиметровых и субмиллиметровых обзоров возникает и задача поиска скоплений галактик с помощью СЗ-эффекта, в которых находятся радиоисточники на малых и больших красных смещениях.}

Ранее мы уже использовали подход, основанный на селекции кандидатов в скопления с помощью СЗ-эффекта в направлении на радиоисточники (см. \cite{rc_planck,wenss_sz,traj_zaporozhets}). В этой работе мы развиваем направления, связанные с ускоренной методикой селекции СЗ-объектов, на основе метода машинного обучения (см. описание первого опыта в \cite{oron_habr}).

В последнее время в астрономических исследованиях все чаще применяются методы глубокого машинного обучения. Например, при изучении распределения пыли и ее свойств в галактике \sout{, используя 6 полос в дальнем инфракрасном диапазоне} \textcolor{red}{} \cite{Dobbels}, при поиске связей между звездами в центре галактики и морфологией самой галактики (см. работу~\cite{Tacchella}), для определения параметров галактик, таких как плотность, металличность, \sout{плотность столбцов}\textcolor{red}{поверхностная плотность} и \textcolor{red}{степени ионизации}\sout{параметр ионизации} по эмиссионным линиям спектров галактик в оптическом, ультрафиолетовом, инфракрасном и субмиллиметровом излучениях линий (см. работу~\cite{Ucci}). Также методы машинного обучения \sout{применяются}\textcolor{red}{применялись} для изучения спектров двух миллионов галактик из Sloan Digital Sky Survey, и в исследованиях 400 галактик с высокими показателями активности джетов \cite{Baron}. Похожие алгоритмы применяются и при классификации изображений протяженных источников\sout{, сделанных} в радиодиапазоне, при изучении радиогалактик на основе морфологических свойств с использованием сверточных нейронных сетей \cite{Aniyan}, при спектральной классификации галактик \cite{Tao}. Наконец, отметим и построение системы машинного обучения при поиске \textcolor{blue}{СЗ-}эффекта по данным HFI Planck \cite{Bonjean2019}, где по высокочастотным данным отобраны ~18\,тыс. кандидатов в СЗ-объекты. \textcolor{blue}{В данной работе представлен алгоритм глобального машинного обучения, позволяющий детектировать кандидаты в скопления галактик на мультичастотных картах излучения по данным миссии Planck с помощью  СЗ-эффекта. В отличие от ранее разработанных методик \cite{Bonjean2019, Melin, Herranz}, в данной работе полнота получаемой выборки анализируется с помощью сравнения с простыми алгоритмами поиска объектов (MM1,MMF3).}

\sout{Задача этой работы --- разработать алгоритм поиска кандидатов в скопления галактик по выборке изображений на частотах HFI диапазона миссии Planck, в областях неба, где могут быть расположены скопления галактик.} \textcolor{red}{Оценивается качество созданной модели, и влияние вариации соотношения сигнал-шум на результат обучения сети, что может быть полезным при работе с данными более низкого качества.} В Разделе 2 описан формат данных, использованных для обучения нейронной сети, методы их получения и обработки. В Разделе 3 \textcolor{red}{представлен непосредственно алгоритм и примеры его применения на данных без шума.} \sout{посвящен нейронным сетям, как методу решения задачи поиска скоплений галактик, и результату его работы на данных без шума.} В Разделе 4 приведены результаты работы сети для случайных гауссовых карт с различным отношением \sout{сигнал/шум}\textcolor{red}{сигнала к шуму}. В заключительном разделе сформулированы выводы о проведенной работе и перспективы её развития.

\section{\sout{Данные и их предобработка} Подготовка данных}
\label{sec:data}
\textcolor{blue}{В работе используются данные, взятые из архивов наблюдений телескопа Planck \footnote{\tt https://pla.esac.esa.int/\#home}. Эти данные были переведены в формат GLESP. Программный пакет GLESP~\cite{glesp1}, использует одноименные схемы пикселизации карт космического микроволнового фона, основываясь на применении нулей полиномов Гаусса--Лежандра для организации сетки разбиения неба, и позволяет получить строгое ортогональное разложение карт. Пакет имеет процедуры для работы с отдельными изображениями на площадках заданного размера  и позволяет быстро преобразовывать температуры в координатах в температурный спектр, то есть производить разложение по сферическим функциям с использованием специального метода интегрирования Гаусса. Для создания выборки объектов с СЗ-эффектом отбирались те карты, для которых заведомо известно, что эффект присутствует. Карты без СЗ-эффекта выбирались по случайным координатам в области, для которой известно, что этого эффекта нет. Подробнее возможности GLESP описаны в работах \cite{glesp1,glesp3}. Данные о полученной выборке объектов находятся в открытом доступе в архиве САО РАН \footnote{\tt http://sed.sao.ru/vo/planck\_maps/}}. 

\textcolor{blue}{Выборка объектов с СЗ--эффектом производилась по трем каталогам: ABELL (Abell Clusters) \cite{Abell}, PSZ1 \cite{Planck1}, а так же был использован второй архив данных Planck --- PSZ2 \cite{Planck2}. Области, в которых нет СЗ--эффекта выбирались в случайных полях на глаз. Для обучения сети было составлено пять каталогов изображений с CЗ-эффектом, каждый из которых состоит из $1 000$ объектов с данными на частотах 100, 143, 217, 353, 545\,ГГц.}
\sout{Телескоп Planck запущен 14 мая 2009 года и успешно выполнял задачи миссии на протяжении 30 месяцев: осуществил пять съемок неба с помощью высокочастотного (High Frequency Instrument, далее --- ``HFI'') и низкочастотного приборов (Low Frequency Instrument, далее --- ``LFI''). Архив данных, полученных в результате миссии Planck, содержит карты компонентов излучения и списки обнаруженных объектов, включая скопления галактик с CЗ-эффектом. Предварительные оценки показывают, что количество объектов с CЗ-эффектом может возрасти до десятков тысяч \cite{Planck1, Planck2}.}

За счёт характерной особенности СЗ-эффекта --- отсутствия сигнала на \textcolor{blue}{на 217 ГГц и ниже}\sout{217 ГГц} --- можно идентифицировать кандидаты в скопления галактик и отличить их от \sout{наблюдаемого} сигнала других радиоисточников \sout{космических источников} \textcolor{blue}{, у которых такого эффекта не наблюдается \cite{Planck2}}. \sout{Пример зависимости интенсивности радиоизлучения от наблюдаемой частоты в объекте с СЗ-эффектом приведен на Рис.~\ref{fig:freq}. На левой панели схемы показаны наблюдательные данные скопления галактик Abell 2313, полученные с телескопа Plank (см. работы \cite{Abell, Planck1, Planck2}. На правой --- показан график. Графики справа от изображения объекта на различных частотах, иллюстрируют зависимость интенсивности излучения от частоты наблюдения.}

%\begin{figure}[h!]
%\onelinecaptionstrue
%\setcaptionmargin{5mm}
%\includegraphics[width=0.45\linewidth]{figures/1567219072543%-Planck_SZEffect_origsize_Xb-2.png}
%\includegraphics[width=0.45\linewidth]{figures/1567219072543%-Planck_SZEffect_origsize_Xb-3.png}
%\includegraphics[width=0.45\linewidth]{figures/1567219072543%-Planck_SZEffect_origsize_Xb-4.png}
%\includegraphics[width=0.45\linewidth]{figures/1567219072543%-Planck_SZEffect_origsize_Xb-5.png}
%\caption{Результаты наблюдений скопления галактик Abell 2313 %миссии телескопа Planck\textcolor{blue}{, картинка взята из %результатов коллаборации Planck ESO для демонстрации эффекта %СЗ. Цветные картинки иллюстрируют излучение источника на %частотах 100, 143, 217 и 353\,ГГц соответственно. Графики %справа показывают зависимость интенсивности излучения от %частоты наблюдения.}}
%\label{fig:freq}
%\end{figure}

\sout{На основе результатов миссии Planck было составлено пять каталогов изображений с CЗ-эффектом, каждый из которых состоит из 1000 наиболее контрастных изображений на частотах 100, 143, 217, 353, 545 ГГц соответственно \cite{planck_zs} для обучения сети.
Каталоги изображений были составлены с помощью пакета GLESP~\cite{glesp1}, использующего одноименную схемы пикселизации карт космического микроволнового фона, основанная на применении нулей полиномов Гаусса--Лежандра для организации сетки разбиения неба и позволяющая получить строгое ортогональное разложение карт. Пакет имеет процедуры для работы с отдельными изображениями на площадках заданного размера.
В GLESP} \textcolor{blue}{Помимо того, что программный код GLESP был использован для преобразования наблюдений телескопа Plank, им также} были вырезаны зоны размером $30\arcmin\times30\arcmin$ в окрестности \sout{радиоисточников}\textcolor{red}{объектов} с CЗ-эффектом \sout{на карте реликтового излучения} на частотах \sout{картах} 100, 143, 217, 353 и 545\,ГГц. \sout{Также для обучения сети мы использовали многочастотные изображения в 1000 случайных полях, отличных по координатам от СЗ-объектов, на картах диапазона HFI.}

\sout{Таким образом}\textcolor{red}{Полученный таким образом}, наш набор данных\sout{ (датасет)} состоит из двух классов\sout{, по 1000 объектов в каждом}\textcolor{red}{: изображения с СЗ-эффектом и без него}. Каждый \sout{объект представляет собой группу из}\textcolor{red}{класс содержит 1000 объектов, которые представляют собой группы из} 5 изображений. 

\sout{Размер полученных изображений 400x400 пикселов. Однако после анализа изображения стало ясно, что исходное разрешение данных ниже, и картину можно без потерь уменьшить до размера 96х96. Уменьшая картинку, мы упрощаем задачу для нейрной сети: в большой картинке, увеличенной из маленькой, локальные признаки не имеют смысла. Уменьшать картинку нужно методом ближайшего соседа, чтобы не потерять  точную информацию интенсивности из-за размытая. Так же было принято решение сделать круглую маску объекта, чтобы быть уверенными, что сеть использует только те данные, которые на самом деле должны влиять на ответ.}

\textcolor{red} {Для каждого объекта мы подготавливаем изображения участков неба размером 30 угловых минут. При перенесении изображения на новую сетку пикселей мы присваиваем каждому узлу сетки значение ближайшего узла на старой сетке (т.н. интерполяция методом ближайшего соседа). Новая сетка нужна, чтобы уменьшить размер изображения, так как это тратило много ресурсов. Обучение на изображениях большего размера нецелесообразно, так как увеличенное изображение не несёт большего количества информации. Применялась круглая маска вокруг объектов как с СЗ-эффектом, так и без СЗ-эффекта, на рис.~\ref{fig:freq11} показан пример объекта PSZ2 G008.94-81.22  из каталога Planck \cite{Planck1} с СЗ-эффектом.} 

\begin{figure}
a) 100 \includegraphics[width=0.14\linewidth]{008_94-81_22_100_gr.png}
143 \includegraphics[width=0.14\linewidth]{008_94-81_22_143_gr.png}
217 \includegraphics[width=0.14\linewidth]{008_94-81_22_217_gr.png}
353 \includegraphics[width=0.14\linewidth]{008_94-81_22_353_gr.png}
545 \includegraphics[width=0.14\linewidth]{008_94-81_22_545_gr.png}\\
б) 100 \includegraphics[width=0.14\linewidth]{0003_100_gr.jpg}
143 \includegraphics[width=0.14\linewidth]{0003_143_gr.jpg}
217 \includegraphics[width=0.14\linewidth]{0003_217_gr.jpg}
353 \includegraphics[width=0.14\linewidth]{0003_353_gr.jpg}
545 \includegraphics[width=0.14\linewidth]{0003_545_gr.jpg}
\caption{\textcolor{blue} {Пример объектов выборки для обучения нейронной сети, соответственно на 100, 143, 217, 353 и 545 ГГц. На рисунках а) представлен объект PSZ2 G008.94-81.22 с СЗ-эффектом, на рисунках б) участок неба без СЗ-эффекта.}}
\label{fig:freq11}
\end{figure}


Мы разделили всю выборку на обучающую, валидационную и тестовую в соотношении 70/15/15, (выбрав в каждом классе 150 случайных объектов в валидационную выборку и 150 -- в тестовую). Таким образом, все три получившиеся выборки остались сбалансированными. Для расширения обучающей выборки мы использовали аугментации (небольшие трансформации)  исходных объектов: поворот на случайный угол и небольшие смещения по горизонтали и вертикали. Важно, чтобы небольшие трансформации были одинаковыми для всех частот одного объекта.

\section{ОПИСАНИЕ АНАЛИТИЧЕСКОГО АЛГОРИТМА}
\label{sec:algorithm}

\sout{В нашей выборке довольно мало данных, поэтому мы должны особенно внимательно следить за переобучением. Переобучение --- это стандартная проблема в машинном обучении, когда алгоритм машинного обучения обучается не общим принципам данных, а признакам, присущим только обучающей выборке. Более простые модели обычно меньше подвержены переобучению. Определяемые зависимости могут быть совсем простыми, и нам помогло бы, если бы сеть сама могла при необходимости упрощать свою архитектуру во время обучения.}

\sout{Нашим  требованиям отвечает архитектура нейросети ResNet18 \cite{resNET1}.}

\sout{Наша задача --- бинарная классификация. Выходом нейросети будет одно число, которое после преобразования  сигмоидальной функцией будет представлять собой ``вероятность'' наличия эффекта. Несмотря на то, что результат сигмоиды всегда принадлежит интервалу [0, 1], он не является на самом деле  отражением эмпирической вероятности наличия эффекта. Для того, чтобы  оно стало вероятностью, его нужно откалибровать. Наша задача проще ---  мы хотим классифицировать объекты на два класса. Для этого нам не нужно  калибровать все пространство ответов, нужно только определить порог, по  которому мы будем бинаризовать полученные ``вероятности''. Порог определяется по наилучшему качеству на валидационной выборке.}

\textcolor{blue}{Мы решаем задачу бинарной классификации. Одной из первых классификационных нейросетей сетей была AlexNet~\cite{AlexNET1}, развитием которой  стала архитектура VGG~\cite{VGG1}. Обе архитектуры используют блоки свёрточных слоёв и пулинга, т.е. постепенное уменьшение пространственного разрешения. За ними следует несколько полносвязных слоёв. Авторы условно обозначают, что свёрточные блоки отвечают за извлечение признаков, в то время как цель полносвязных слоев --- комбинирование извлечённых признаков для получения ответов. Эти архитектуры имеют несколько ограничений. Во-первых, полносвязные слои содержат очень много параметров и являются самой вычислительно сложной частью моделей. Во-вторых, более глубокие архитектуры на основе таких моделей не обучаются из-за затухания градиента.}

\textcolor{blue}{Решением этих проблем стали архитектуры ResNet~\cite{resNET1} и InceptionNet~\cite{InceptionNet1}. ResNet использует ``обходные пути'' (residual connections) в каждом свёрточном блоке, что позволяет градиенту течь через них и не затухать. В каждом блоке есть обучающийся параметр, который регулирует, какая доля сигнала проходит через обходной путь. Сеть имеет возможность сойтись к решению, при котором в части блоков весь сигнал пойдет через обходной путь, то есть часть блоков будет отключена. Благодаря "обходным путям" удалось обучить очень глубокие ResNet-модели (до 152 слоёв)~\cite{resNET1}. InceptionNet использует промежуточные выходы из модели, что также помогает избежать затухания градиентов во время обучения. Дальнейшее развитие архитектур разделилось на два вектора --- модели, дающие наибольшее качество классификации, и модели, вычисляющиеся \sout{эффективно}\textcolor{red}{быстро} (например, MobileNet~\cite{MobileNet}).}

\textcolor{blue}{Из-за небольшого количества данных в обучающей выборке мы не имеем возможности использовать глубокие архитектуры, нацеленные на достижение максимального качества ценой тяжелых вычислений. Мы обучили MobileNet, VGG, и в итоге остановились на ResNet18. Эта сеть показывает качество, сравнимое с другими моделями, но в то же время дает большую стабильность метрик качества при различном разбиении выборки на обучающую и тестовую. Мы предполагаем, что такая архитектура более удачно подходит для этой задачи. Мы не приводим подробного сравнения результатов на разных архитектурах, так как основной нашей задачей является исследование принципиальной возможности использования нейросетевого подхода для детекции объектов.}

\textcolor{blue}{В качестве программного инструмента для обучения применяется библиотека PyTorch~\cite{NEURIPS2019_9015} без предобученных весов. Наша сеть принимает пятиканальные изображения, что делает нецелесообразным перенос весов от сети, обученной на трёхканальных (RGB) изображениях из ImageNet \cite{Russakovsky}.}
В качестве функции ошибок \sout{используем }\textcolor{red}{используется} бинарн\sout{ую}\textcolor{red}{ая} кросс-энтропи\sout{ю}\textcolor{red}{я}. Нейросеть обучается с помощью стохастического градиентного спуска, оптимизатором RAdam \cite{liu2019radam}. \textcolor{blue} {Мы использовали стандартные параметры RAdam, кроме $\text{learning rate} = 2\cdot10^{-4}$. Этот параметр подбирался по валидационной выборке.}


\section{РЕЗУЛЬТАТЫ}
\label{sec:results}

Для оценки \sout{качества}\textcolor{red}{качества результатов} созданной нами модели необходимо оценить её точность. Для этого существуют метрики, к которым относятся достоверность, точность и полнота, F-мера~\cite{Hossin2015}. Поскольку классы сбалансированы, мы можем использовать простейшую метрику accuracy --- долю правильных ответов.

\sout{Ставим эксперимент}\textcolor{red}{Эксперимент поставлен} следующим образом:
\begin{enumerate}
    \item \sout{Обучаемся на обучающей выборке}\textcolor{red}{Обучение нейросети на обучающей выборке в течение} 150 эпох (параметр ``эпоха'' характеризует уровень натренированности нейросети --- количество проходов через всю обучающую выборку), \sout{смотрим на}\textcolor{red}{после которого сравниваются} графики функции потерь и accuracy. \sout{Сохраняем}\textcolor{red}{Выбираются те} веса модели, для которых функция потерь на валидационной выборке показывает наименьшее значение;
    \item Для лучшей \sout{модели определяем лучший}\textcolor{red}{из моделей определяется наиболее подходящий} порог бинаризации по accuracy на валидационной выборке;
    \item Для лучшей модели и лучшего порога бинаризации считаем accuracy на тестовой выборке\textcolor{blue}{, которая будет служить мерой качества работы модели.}
\end{enumerate}


Для сравнения мы также протестрировали возможность классификации с помощью подхода, не включающего нейронные сети. \textcolor{blue}{В этом случае требуется подготовить признаки самостоятельно. В качестве признаков для каждого канала были выбраны ``среднее'', ``дисперсия'', а также гистограмма из 5 бинов с фиксированными порогами. Используется модель ансамбль деревьев (RandomForest). }
\sout{Обучили модель градиентного бустинга (далее GB) на признаках ``среднее'' и ``дисперсия'' для каждого из 5 изображений (всего 10 признаков на объект). Мы не занимались основательно генерацией признаков и подбором параметров.} Цель этого эксперимента --- показать, что задача не решается ``в лоб'' и использование нейросетей в поиске объектов с СЗ-эффектом является обоснованным. \sout{Получившиеся р} \textcolor{red}{Р}езультаты \textcolor{blue}{представлены} в таблице~\ref{table:comparison}.


\begin{table}[!h]
\onelinecaptionstrue
\setcaptionmargin{5mm}
\begin{tabular}{|l|l|l|l|}
\hline
\multirow{2}{*}{Алгоритм} & \multirow{2}{*}{Параметры} & 
\multicolumn{2}{c|}{Accuracy} \\
 &  & Валидация & Тест \\
\hline
\sout{GB} & \sout{depth=5, lr=0.01, 117 итераций} & \sout{0.690} & \sout{0.653} \\
\sout{ResNet18} & \sout{threshold=0.32} & \sout{\textbf{0.927}} & \sout{ \textbf{0.880}} \\
\hline
\end{tabular}
\caption{\sout{Сравнение качества классификации объектов обученными моделями на валидационной и тестовой выборках}}
\label{table:comparison}
\end{table}


\begin{table}[!h]
\onelinecaptionstrue
\setcaptionmargin{5mm}
\textcolor{red}{\begin{tabular}{|l|l|l|l|}
\hline
Алгоритм & Параметры & Accuracy & Recall \\
\hline
MMF1 & & - & 0.747 \\
MMF3 & & - & 0.740 \\
PwS &  & - & 0.620 \\
% \sout{GB} & \sout{depth=5, lr=0.01, 117 итераций} & \sout{0.690} & \sout{0.653} \\
RandomForest & n\_estimators=300, max\_depth=10 & 0.666 & 0.660 \\
ResNet18 & threshold=0.4 & \textbf{0.801} & \textbf{0.813} \\
\hline
\end{tabular}
\caption{Сравнение качества классификации объектов различными моделями на валидационной и тестовой выборках. ResNet18 --- это метод, используемый в данной работе.}
\label{table:comparison}}
\end{table}


\textcolor{blue}{Мы провели сравнение по полноте с ранее приведёнными простыми методами ММF1и MMF3 \cite{Melin, Herranz}, PwS. Полнота (Recall) \sout{ -- это сколько объектов мы угадали и присутствует ли они в выборках этих алгоритмов. Этот процесс заключается в том, что у нас есть 1000 объектов, в который 100\%} \textcolor{red}{характеризует количество правильно идентифицированных объектов и их присутствие в выборке этих алгоритмов. Имеется выборка из 1000 объектов, в которых заведомо} присутствует эффект-СЗ. После обучения алгоритма мы подбираем порог по accuracy, руководствуясь полноценной выборкой. Сравнивать наши данные с простыми метриками мы можем только по полноте, сравнение происходит при неизменном значении порога. Результаты сравнения показаны в табл.~\ref{table:comparison}, которая демонстирирует более высокое качество классификации при использовании нашей модели (более подробное описание см. в Приложении).}

Для проверки \sout{эффективности}\textcolor{red}{быстроты} обнаружения объектов по предложенной методике с помощью пакета GLESP \cite{glesp1} мы сгенерировали случайные гауссовы карты с различным отношением сигнал/шум (S/N) для объектов \sout{типа Сюняева-Зельдовича}\textcolor{red}{с СЗ-эффектом}.
Для построения моделей мы использовали приближения к СЗ-сигналу к коэффициентами $k_{100}=-0.4$, $k_{100}=-0.4$, $k_{143}=-0.5$, $k_{217}=0.0$, $k_{353}=1.0$, $k_{545}=0.5$ для частот 100, 143, 217, 353 и 545\,ГГц, соответственно. Относительные коэффициенты рассчитаны по данным наблюдений скопления Abell\,2319 \cite{Abell, Planck1, Planck2}. Модельный источник генерировался с шириной по половинной мощности равной $5\arcmin$. Отношение $\text{S}/\text{N}$ вычислялось для сигнала на частоте 353\,ГГц и \sout{соответственно}\textcolor{red}{масштабировалось} пропорционально на других частотах HFI. 


Всего было сгенерировано 100 моделей радиоисточников. Во всех сгенерированных объектах \textcolor{red}{СЗ-}эффект должен наблюдаться. \sout{Поэтому для оценки качества здесь будем использовать другую метрику --- полноту (recall).}
На рис.~\ref{fig:sim} приведены \sout{результаты}\textcolor{red}{результаты сравнения качества (полноты) при различных соотношениях сигнал/шум}. 

\begin{figure}[h!]
\onelinecaptionstrue
\setcaptionmargin{5mm}
\includegraphics[width=0.5\linewidth]{figures/sim_recall_linscale.eps}
\caption{Качество обнаружения эффекта (полнота) на модельных данных при разных соотношениях сигнал/шум}
\label{fig:sim}
\end{figure}

Как видно из рис.~\ref{fig:sim}, полнота ответов модели (recall) растет с упрощением задачи. При соотношении сигнал/шум $=1$ сеть предсказывает, что на всех объектах нет эффекта. С усилением сигнала увеличивается уверенность сети, на соотношениях сигнал/шум $=5$ и больше сеть классифицирует все объекты \sout{в правильный класс}\textcolor{red}{правильно}.


\section{ЗАКЛЮЧЕНИЕ}
\label{sec:conclusion}

Мы построили \sout{процедуру обнаружения}\textcolor{red}{модель поиска} СЗ-эффекта. \sout{Пройден путь от описания влияния СЗ-эффекта до применения его в качестве инструмента поиска галактик и их скоплений на подготовленных изображениях в разных частотных диапазонах.} Было показано, что для достижения лучшего качества распознавания необходимо использовать нейросетевые модели.

Мы показали, что данных подход работает и может быть применен к анализу \sout{больших} массивов данных \textcolor{blue}{для поиска радиоисточников с эффектом СЗ и случайные области неба, где предположительно нет СЗ-эффекта. Реализованный подход дополняет существующие алгоритмы \cite{Bonjean2019} поиска таких объектов, позволяя работать с данными в формате пакета GLESP.} \sout{Наши процедуры многоступенчатого анализа, сочетающие построение изображений в окрестности радиоисточников, селекцию объектов и подсчеты фоновых оптических объектов в поле радиоисточника с подключением сравнения с число в среднем случайно поле, позволит по ускоренной процедуре отбирать наиболее вероятные кандидаты в скопления галактик. Причем с учетом возможностей селекции радиоисточников с большими красными смещениями методика позволит искать далекие скопления галактик, что делает ее пока уникальной.}

Развитие работы предполагает создание конвейерного подхода \sout{поиска}\textcolor{red}{к поиску} СЗ-объектов по данным имеющихся каталогов радиоисточников (таких как WENSS \cite{Rengelink}, NVSS \cite{Condon} и других), а также модификацию и развитие алгоритмов селекции. Отметим некоторые \sout{моменты}\textcolor{red}{особенности} развития подхода.

\begin{itemize}
    \item Выборка не обязана быть сбалансированной. Ее можно увеличить, включив в набор менее контрастные из $\sim$1600 объектов для \sout{таргет класса}\textcolor{red}{класса объектов с СЗ-эффектом}, а также дополнительно и\sout{х}\textcolor{red}{з} специализированных обзоров, \sout{вроде}\textcolor{red}{например,} \cite{act_zs}. Это поможет обучить алгоритм на ``неразличимых'' подвыборках. Кроме того, можно \sout{значительно} расширить \sout{негативный} класс \sout{объектов}\textcolor{red}{объектов без СЗ-эффекта} в десятки и сотни раз, что является неопасн\sout{о}\textcolor{red}{ы}м перекосом в данных для \textcolor{blue}{нейро}сети.
    \item Дополнительный способ расширить обучающую выборку --- увеличить число смоделированных объектов. \sout{Они проще, но, если они передают ту же самую общую идею, что и основной датасет --- они могут быть полезны}\textcolor{red}{Они не в полной мере воспроизводят сложность реальных объектов, однако могут быть полезны, если учитывают основные проявления эффекта.} \sout{Если готовится модель для нахождения новых объектов (то есть в применении будут только реальные  данные), то} \textcolor{
    red}{В случае применения модели для поиска новых, ранее не обнаруженных объектов в реальных наблюдательных данных} добавление смоделированных \sout{данных}\textcolor{
    red}{объектов} в \sout{обучение}\textcolor{red}{обучающую выборку} может улучшить  качество модели.
    \item \sout{Подключить}\textcolor{red}{Можно рассмотреть способ подключить} варьирование цветной нормировки изображений для определения оптимального подхода при анализе данных с СЗ-эффектом.
    \item Подключить \sout{возможно}\textcolor{red}{возможность} подачи на вход сети при обучении не только исходных данных, но и реалистично измененных в рамках афинных преобразований. Это поможет сетке выучить общие принципы и не переобучаться на \sout{семплы}\textcolor{red}{определённые образцы из} обучающей выборки.
\end{itemize}

{\small
{\bf БЛАГОДАРНОСТИ}

Авторы выражают свою признательность Фонду поддержки научных, культурных и образовательных инициатив ``Траектория'', В.~С.~Ивашкину за содействие в проведении данной работы. 
\sout{обработки данных в Planck Legacy Archive.}
В работе  использован пакет GLESP \footnote{\tt http://www.glesp.nbi.dk} для анализа протяженного излучения на сфере. 
Авторы выражают благодарность рецензенту за полезные замечания, которые привели к коррекции текста и улучшению его понимания.}
\selectlanguage{russian}


\begin{thebibliography}{99}

\bibitem{zs}
Ya.~B.~Zeldovich and R.~A.~Sunyaev, Astrophys.~Sp. Sci. {\bf 4}, 301 (1969).

\bibitem{Allen} S.~W.~Allen, A. E. Evrard, and A. B. Mantz. Annual Review of Astronomy and Astrophysics, {\bf 49}, 409 (2011).

\bibitem{Kravtsov} A.~V.~Kravtsov and S.~Borgani. Annual Review of Astronomy and Astrophysics, {\bf 50}, 353 (2012).

\bibitem{Nagai} D.~Nagai. In American Institute of Physics
Conference Series, {\bf 1632}, 88 (2014).

\bibitem{Sarazin} C.~L.~Sarazin. Reviews of Modern Physics, {\bf 58}, 1 (1986).

\bibitem{Kormendy} J.~Kormendy and S.~Djorgovski.Annual Review of Astronomy and Astrophysics, {\bf 27}, 235 (1989).


\bibitem{Bykov} A.~M.~Bykov, H.~Bloemen, and Y.~A.~Uvarov. Astronomy and Astrophysics, {\bf 362}, 886 (2000).

\bibitem{Basu} K.~Basu, J.~Erler, J.~Chluba et al. Bulletin of the American Astronomical Society, {\bf 51}, 302 (2019).

\bibitem{planck_zs}
Planck Collaboration: P.~A.~R.~Ade, N.~Aghanim, M.~Arnaud, M. et al., Astronomy and Astrophysics, {\bf 594}, 19 (2016).

\bibitem{Barbosa1996}
D.~Barbosa, J.~G.~Bartlett, A.~Blanchard, et al., Astron. Astroph., {\bf 314}, 13 (1996).

\bibitem{2009MNRAS.397L..41C}
D.~Cunnama, A.~Faltenbacher, C.~Cress, et al., MNRAS, {\bf 397}, L41 (2009).

\bibitem{dsol_model}
 D.~I.~Solovyov, O.~V.~Verkhodanov., Astrophys.~Bull., {\bf 72}, 217 (2017).
 
\bibitem{spt_zs}
K. Vanderlinde et al., ApJ, arXiv:1003.0003, {\bf 722}, 1180 (2010).

\bibitem{act_zs}
M.Hasselfield, M.Hilton, T.A.Marriage., JCAP, arXiv:1301.0816, {\bf 07}, 008 (2013).

\bibitem{Abell}
G.~O. Abell, H.~G. Corwin and  R.~P. Olowin., Astrophysical Journal Supplement {\bf 70}, 1 (1989).


\bibitem{Planck2}
Planck Collaboration, Planck 2018 results, VI. Cosmological parameters,
arxiv:1807.06209, (2018). 


\bibitem{Planck1}
Planck Collaboration, Planck 2018 results, I. Overview and the cosmological legacy of
Planck, arxiv:1807.06205, (2018).


\bibitem{rg_list1}
M.~L.~Khabibullina and O.~V.~Verkhodanov, Astrophys. Bull. {\bf 64}, 123 (2009).

\bibitem{rg_book}
Верходанов О.~В.~Парийский Ю.~Н., Радиогалактики и космология. (М.:Физмалит, 2009).

\bibitem{blum_miley} G.~Blumenthal and G.~Miley, \aaa {\bf 80}, 13 (1979).

\bibitem{par_big3a}
 Yu.~N.~Parijskij,  W.~M.~Goss,  A.~I.~Kopylov et al., Bull. SAO, {\bf 40}, 5 (1996).

\bibitem{uss_list} C.~de~Breuck,   W.~van~Breugel,  H.~J.~A.~R\"ottgering and G.~Miley, \aas, {\bf 143}, 303 (2000).

\bibitem{rg_protocl}
B.~P.~Venemans, H.~J.~A.~Rottgering, G.~K.~Miley,  et al., Astron. Astrophys.,  {\bf 461}, 823 (2007).

\bibitem{tim_clust}
T.~V.~Keshelava and O.~V.~Verkhodanov., Astrophys. Bull., {\bf 70}, 257 (2015).


\bibitem{rc_planck}
O.~V.~Verkhodanov, E.~K.~Majorova, O.~P.~Zhelenkova, et al., Astrophys. Bull., {\bf 70}, 156 (2015).

\bibitem{wenss_sz}
O.~V.~Verkhodanov, N.~V.~Verkhodanova, O.~S.~Ulakhovich, et al., Astrophys. Bull., {\bf 73}, 1 (2018).

\bibitem{traj_zaporozhets}
A.~A.~Zaporozhets, O.~V.~Verkhodanov, Astrophys. Bull. {\bf 74}, 247 (2019).

\bibitem{oron_habr}
A.~Oronovskaya, {\tt https://habr.com/ru/users/sunny\_space/posts/}, (2018).

\bibitem{Dobbels} 
W.~Dobbels, M.~Baes, S.~Viaene, et al., arXiv:1910.06330, (2019).

\bibitem{Tacchella}
S.~Tacchella, B.~Diemer, L.~Hernquist, et al., MNRAS {\bf 487}, 5416 (2019).

\bibitem{Ucci}
G.~Ucci, A.~Ferrara, S.~Gallerani, et al., MNRAS {\bf 465}, 1144 (2017).

\bibitem{Baron}
D.~Baron, D.~Poznanski, MNRAS {\bf 465}, 4530 (2017).

\bibitem{Aniyan}
A.~K.~Aniyan, K.~Thorat, ApJS {\bf 230}, id. 20 (2017).

\bibitem{Tao}
Y.~Tao, Y.~Zhang, C.~Cui, et al., Astr. Data Analysis Soft. and Syst. XXVII, {\bf 523}, (2018).

\bibitem{Bonjean2019}
V.~Bonjean, Astronomy and Astrophysics, {\bf 634}, 11 (2020).


\bibitem{Melin} 
J.~BMelin, J.~G.~Bartlett and J.~Delabrouille, A\&A, {\bf 459}, 3411 (2006).

\bibitem{Herranz} D.~Herranz, J.~L.~Sanz, M.~P.~Hobson et al., MNRAS, {\bf 336}, 1057 (2002).

\bibitem{glesp1}
A.G.Doroshkevich, P.D.Naselsky, O.V.Verkhodanov, et al., Int. J. Mod. Phys. D {\bf 14}, astro-ph/0305537, 275 (2003).


\bibitem{Hossin2015}
M.~Hossin, M.N.~Sulaiman, Internat. J. of Data Mining and Knowledge Manag. Proc., {\bf 5}, 1 (2015).

\bibitem{glesp3}
A.~G.~Doroshkevich, O.~V.~Verkhodanov, P.~D.~Naselsky, et al.,{Int. J. Mod. Phys.} D {\bf 20}, arXiv:0904.2517, 1053 (2011).


\bibitem{AlexNET1}
A.~Krizhevsky, I.~Sutskever, G.~E.~Hinton, In Advances in neural information processing systems, 1097 (2012).

\bibitem{VGG1}
K.~Simonyan and A.~Zisserman, arXiv preprint, arXiv:1409.1556, (2014).


\bibitem{resNET1}
K.~He, X.~Zhang, S.~Ren, S., et al., arXiv e-prints, arXiv:1512.03385, 770 (2015).

\bibitem{InceptionNet1}
C.~Szegedy, W.~Liu, Y.~Jia, P.~Sermanet et al., IEEE Conf. on Comp. Vision and Pattern Recog., INSPEC Accession Number: 15523970, DOI: 10.1109/CVPR.2015.7298594, (2015).

\bibitem{MobileNet}
A.~G.~Howard, M.~Zhu, B.~Chen et al., arXiv preprint arXiv:1704.04861, (2017).


\bibitem{NEURIPS2019_9015}
A.~Paszke, S.~Gross, F.~Massa, et al.,  arXiv:1912.01703, (2019). 

\bibitem{Russakovsky}
O. Russakovsky, J. Deng, H. Su et al., arXiv e-prints, arXiv:1409.0575, (2014).

\bibitem{liu2019radam}
L.~Liu, H.~Jiang, P.~He, W.~Chen et al., Published as a conf. paper at ICLR, arXiv:1908.03265, (2020).

\bibitem{Rengelink} R.~B. Rengelink, Y. Tang,  A.~G. de Bruyn et al., Astron. and Astroph. Sup. {\bf 124}, 259 (1997).

\bibitem{Condon} J.~J. Condon, W.~D. Cotton, E.~W. Greisen et al.,
 The Astronomical Journal {\bf 115}, 1693 (1998).
 
\end{thebibliography}


\section{ПРИЛОЖЕНИЕ}
\label{sec:conclusion1}

\textcolor{blue}{В основе работы лежат метрики, которые работают с двумя классами. Назовем те объекты, в которых есть СЗ-эффект, positive классом (P), а те, в которых нет --- negative (N). Будем называть ответы, которые мы знаем заранее, ground truth, (gt), а ответы, которые мы предсказываем --- prediction, (pred).}

\textcolor{blue}{Рассмотрим сначала один объект выборки. Мы знаем заранее правильный ответ ($P$ или $N$), и мы предсказываем какой-то ответ ($P$ или $N$). Есть 4 варианта:}
\textcolor{blue}{\begin{itemize}
    \item $gt = P$, $pred = P$ --- здесь СЗ-эффект был на самом деле, и мы его предсказали. Назовем это true positive ($TP$);
    \item $gt = N$, $pred = N$ --- СЗ-эффекта не было, и мы предсказали, что его нет. Это "true negative" ($TN$);
    \item $gt = P$, $pred = N$ --- СЗ-эффект был, но мы его не заметили. Это "false negative" ($FN$);
    \item$ gt = N$, $pred = P$ --- эффекта не было, но мы считаем, что он был. Это "false positive" ($FP$).
\end{itemize}}

\textcolor{blue}{Для списка объектов выборки у нас есть список правильных ответов (например, $gt=PPPNNNN$) и список наших предсказаний (например, $pred=PNNPPPN$). Каждый объект относится к одному из 4 случаев: $TP$, $TN$, $FP$, $FN$.} 

\textcolor{blue}{На основе этих понятий можно сформировать все метрики, которые возможны для двух классов. Нас интересуют две:
\begin{itemize}
    \item  accuracy --- это отношение доли объектов, где правильно угаданы объекты с СЗ-эффектом, ко всей выборке объектов. То есть $TP+TN/(TP+TN+FP+FN)$..
    \item recall --- это сколько из объектов $gt = P$ предсказано как $P$. То есть $TP/(TP+FN)$.
\end{itemize}}

\textcolor{blue}{Метрики accuracy можно продемонстрировать на следующем примере:
\begin{itemize}
  \item  $gt = PPPPNNNN$, $pred = NNNNNNNN$ и accuracy $= 0.5$;
  \item $gt = PPPPNNNN$, $pred = PNPNPNPN$ и accuracy $= 0.5$;
  \item $gt = PNNNNNNN$, $pred = NNNNNNNN$ и accuracy $= 7/8$.
\end{itemize}}
 
\textcolor{blue}{Видно, что для сбалансированных выборок (когда в $gt$ примерно одинаковое количество $P$ и $N$) accuracy для константного предсказания всегда $N$ или случайного значения, дает результат не ниже $0.5$.}
\textcolor{blue}{Но для несбалансированных классов может получиться большее значение, что показывает уже не качество наших предсказаний, а сбалансированность выборки. Короткий итог --- accuracy полезна только в сбалансированных выборках.}

\textcolor{blue}{Границы применимости метрики recall можно продемонстрировать, например, так:
\begin{itemize}
  \item $gt = PPPPNNNN$, $pred = NNNNNNNN$ и recall $= 0$;
  \item $gt = PPPPNNNN$, $pred=PNPNPNPN$ и recall $= 0.5$;
  \item $gt = PPPPNNNN$, $pred=PPPPPPPP$ и recall $= 1$;
  \item $gt = PNNNNNNN$, $pred=PPPPPPPP$ и recall $= 1$.
\end{itemize}}

\textcolor{blue}{Выбить $1$ по recall не так уж сложно --- нужно просто всегда предсказывать $P$. Поэтому смотреть на метрику recall в отрыве от других нельзя --- можно не заметить, как модель вместо умных предсказаний просто выдает $P$.}

\textcolor{blue}{Метрики работают по такому принципу:
\begin{itemize}
\item Наша выборка сбалансирована, поэтому всегда можно использоват accuracy.
\item Если мы сверяемся с каталогами, то получается следующая вещь. В каталогах есть только правильные объекты (TP), и качество каталога мы можем измерить только по отсутствию в нем каких-то объектов (FN). Accuracy требует еще TN и FP, их у нас нет. Остается только использовать recall.
\item Это нормально, считать recall каталогов, потому что они не были заточены на максимизацию recall. 
\end{itemize}}

\textcolor{blue}{Только нам тоже нельзя обучать сетку предсказыванием с как можно большим recall --- это приведет к тому, что мы всегда будет предсказывать $P$. Поэтому мы не старались маскимизировать recall.}

\textcolor{blue}{Наша нейронная сеть в качестве результат выдаёт число от 0 до 1 --- вероятность того, что объект принадлежит класс --- а нам нужен ответ --- P или N. То есть надо установить какой-то порог, выше которого --- P. Если выборка действительно сбалансирована и очень большая, то правильный порог --- 0.5, но в этой работе порог определяется отдельно --- это и есть порог бинаризации. Чтобы определить порог мы делим выборку на три части: }
\textcolor{blue}{\begin{itemize}
\item Train --- учимся, т.е. сдвигаем веса сети так, чтобы минимизировать функцию потерь; 
\item Validation ---подвыборка, по которой подбирается порог, т.е. перебираются пороги от 0 до 1 и выбирается тот, который в среднем на валидационной выборке дает лучшее accuracy;
\item Test --- на этой подвыборке измеряются итоговые accuracy и recall, которые попадают в таблицу для сравнения.
\end{itemize}}

\textcolor{blue}{Если бы мы находили порог по recall, то порог бинаризации был бы 0 --- для всех предсказаний выдавать $P$. Мы победили бы каталоги, но получили бы низкую accuracy, а именно, 0.5, потому что выборка сбалансирована.}


\end{document}
